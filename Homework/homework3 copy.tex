\documentclass[12pt]{article}

\usepackage[margin=1in]{geometry} 
\usepackage{amsmath,amsthm,amssymb,enumitem,bbm,xparse}

\newcommand{\N}{\mathbb{N}}
\newcommand{\Z}{\mathbb{Z}}
\newcommand{\R}{\mathbb{R}}
\newcommand{\Rd}{\mathbb{R}^{d}}
\newcommand{\exr}{[-\infty, \infty]}
\newcommand{\Biggnorm}{\Bigg | \Bigg |}

\NewDocumentCommand\closure{sm}
  {\IfBooleanTF{#1}{\overline{#2}}{\bar{#2}}}

\newenvironment{ex}[2][Exercise]{\begin{trivlist}
\item[\hskip \labelsep {\bfseries #1}\hskip \labelsep {\bfseries #2.}]}{\end{trivlist}}

\newenvironment{sol}[1][Solution]{\begin{trivlist}
\item[\hskip \labelsep {\bfseries #1:}]}{\end{trivlist}}

\newenvironment{theorem}[2][Theorem]{\begin{trivlist}
\item[\hskip \labelsep {\bfseries #1}\hskip \labelsep {\bfseries #2.}]}{\end{trivlist}}
    
\newenvironment{lemma}[2][Lemma]{\begin{trivlist}
\item[\hskip \labelsep {\bfseries #1}\hskip \labelsep {\bfseries #2.}]}{\end{trivlist}}
    
\newenvironment{definition}[2][Definition]{\begin{trivlist}
\item[\hskip \labelsep {\bfseries #1}\hskip \labelsep {\bfseries #2.}]}{\end{trivlist}}
    
\newenvironment{example}[1][Example]{\begin{trivlist}
\item[\hskip \labelsep {\bfseries #1:}]}{\end{trivlist}}
\begin{document}
\noindent David Owen Horace Cutler \hfill {\Large Math 237: Homework 2} \hfill \today

\begin{ex}{1}
    Show that the only function $f \in L^1(\mathbb{R})$ such that $f = f * f$ is $f = 0$ a.e. 
    \begin{proof}
        Let $f \in L^1(\mathbb{R})$ such that $f = f * f$. Recall we have that by Heil Exercise 9.2.6 that then 
        $$\hat{f} = (f * f)\hat{\;} = \hat{f}\hat{f},$$
        so for any particular $\xi$ we have 
        $$\hat{f}(\xi)^2 - \hat{f}(\xi) = 0,$$
        so we consequently have 
        $$\hat{f}(\xi) = 0 \text{ or } 1, \quad \forall \xi \in \mathbb{R}.$$
        Recall then that $\hat{f}$ is continuous on $\mathbb{R}$; it trivially follows that it thus must be either identically 0 or identically 1, as we would otherwise contradict the intermediate value theorem. \\ \\
        However, the Riemann-Lebesgue Lemma guarentees that $|\hat{f}|$ should decay to $0$ as $|x| \rightarrow \infty$; so the only possibility is that $\hat{f}$ is constantly 0. \\ \\
        Note then clearly $\hat{0} = 0$; by the uniqueness of Fourier transforms then, as we thus have $\hat{f} = \hat{0}$, it must be then that $f = 0 \text{ a.e.}$, as desired
    \end{proof}
\end{ex}

\begin{ex}{6}
    Suppose $f \in \text{AC}(\mathbb{T})$, i.e., $f$ is $1$-periodic and absolutely continuous on $[0,1]$.
    \begin{enumerate}[label=6.\arabic*]
        \item Prove that $\hat{f'}(n) = 2\pi in\hat{f}(n)$ for $n \in \mathbb{Z}$ and conclude $\underset{|n| \rightarrow \infty}{\lim} n\hat{f}(n) = 0$.
        \begin{proof}
            By assumption $f \in AC(\mathbb{T})$. We know of course that $e^{2\pi in\xi}$ is furthermore absolutely continuous are we observe it is continuously differentiable. \\ \\
            Thus, we can apply integration by parts, which gets us the following (note $\frac{d}{d\xi} e^{-2\pi in\xi} = -2\pi ine^{-2\pi in\xi}$):
            \begin{align*}
                \hat{f'}(n) = \int_0^1 f'(\xi) e^{-2\pi i n\xi} d\xi \\
                = e^{-2\pi in\xi}f(1) - f(0) - \int_0^1 -2\pi in f(\xi)e^{-2\pi in \xi} d\xi \\
                = e^{-2\pi in\xi}f(1) - f(0) + 2\pi i n \hat{f}(n)
            \end{align*}
            We note then that for any value of $n \in \mathbb{Z}$, we have $e^{-2\pi in\xi} = 1$. Moreover, as $f$ is $1$-periodic, we have $f(0) = f(1)$. Thus we can further reduce 
            \begin{align*}
                e^{-2\pi in\xi}f(1) - f(0) + 2\pi i n \hat{f}(n) \\
                = 2\pi in\hat{f}(n),
            \end{align*}
            as desired. Recall the by the Riemann-Lebesgue Lemma we know 
            $$\underset{|n| \rightarrow \infty}{\lim} |\hat{f'}(n)| = 0,$$
            and thus 
            $$\underset{|n| \rightarrow \infty}{\lim} |\hat{f'}(n)| = \underset{|n| \rightarrow \infty}{\lim} |2\pi i n \hat{f}(n)| = \underset{|n| \rightarrow \infty}{\lim} 2\pi n |\hat{f}(n)| = 0,$$
            where from limit rules multiplying by $\frac{1}{2\pi}$ gets the limit of $n|\hat{f}(n)|$ as $0$, which clearly implies the same for $n\hat{f}(n)$, as additionally desired.
        \end{proof}
        \item Show that if $\int_0^1 f(x) dx = 0$, then 
        $$\int_0^1 |f(x)|^2 dx \leq \frac{1}{4\pi^2} \int_0^1 |f'(x)|^2 dx$$
    \end{enumerate}
\end{ex}

\begin{ex}{12}
    Fix $g \in L^2(\mathbb{R})$. Prove that $\{T_kg = g(\cdot - k)\}_{k \in \mathbb{Z}}$ is an orthonormal sequence if and only if 
    $$\sum_{k \in \mathbb{Z}} |\hat{g}(\xi - k)|^2 = 1 \text{ a.e.}$$
    \begin{proof}
        Let $n, m \in \mathbb{Z}$. We can derive the following:
        \begin{align*}
            \langle T_ng, T_mg \rangle \\
            = \int_{-\infty}^\infty T_ng(x)\overline{T_mg(x)}dx \\
            = \int_{\infty}^\infty (T_ng)\hat{\;}(\xi)\overline{(T_mg)\hat{\;}(\xi)} d\xi && \text{(Parseval's theorem)}
        \end{align*}
        We would like to say now that $(T_mg)\hat{\;}(\xi) = e^{-2\pi i m \xi}\hat{g}(\xi)$ for a.e. $\xi$, i.e. we have the translation identity for $L^2(\mathbb{R})$ functions. 
        \\ \\ For this, define $g_R = \mathbbm{1}_{[-R, R]}g$. Then clearly each $g_R$ is in $L^2(\mathbb{R})$, as as we have $L^2 \subseteq L^1$ on bounded domains we have $g_R \in L^1(\mathbb{R})$, i.e. the Fourier transform is defined. \\ \\
        Note then 
        \begin{align*}
            (T_m g_R)\hat{\;}(\xi) \\
            = \int_{-\infty}^\infty T_mg_R(x)e^{-2\pi i \xi x}dx \\ 
            = \int_{-\infty}^\infty g(x- m)e^{-2\pi i \xi x}dx \\ 
            = \int_{-\infty}^\infty g(u)e^{-2\pi i \xi (u + m)}dx && (u-\text{substitution}) \\
            = \int_{-\infty}^\infty \Big ( g_R(u)e^{-2\pi i \xi u} \Big)e^{-2\pi i \xi m} \\
            = e^{-2\pi i \xi m} \int_{-\infty}^\infty g_R(u)e^{-2\pi i \xi u} \\ 
            = e^{-2\pi i \xi m} \hat{g_R}(\xi),
        \end{align*}
        moreover, note we have $g_R \xrightarrow{L^2} g$. Thus, by definition, for $g \in L^2(\mathbb{R})$ it is precisely $\hat{g} = \underset{R \rightarrow \infty}{\lim} \hat{g_R}$. Thus we get 
        \begin{align*}
            (T_m g)\hat{\;} \\
            = \Big (T_m \Big (\underset{R \rightarrow \infty}{\lim} g_R \Big)\Big)\hat{\;} \\
            = \Big ( \underset{R \rightarrow \infty}{\lim} T_mg_R \Big)\hat{\;} && \text{(Translation commutes with limit)} \\ 
            = \underset{R \rightarrow \infty}{\lim} (T_mg_R)\hat{\;} && \text{(Fourier transform operator is continuous)} \\
            = \underset{R \rightarrow \infty}{\lim} e^{-2\pi i m\xi}(g_R)\hat{\;} \\
            = e^{-2\pi i m\xi} \underset{R \rightarrow \infty}{\lim} \hat{g_R} \\
            = e^{-2\pi i m\xi}\hat{g},
        \end{align*}
        where this equality is in the $L^2$ sense, i.e. precisely that $(T_mg)\hat{\;}(\xi) = e^{-2\pi i m \xi}\hat{g}(\xi)$ for a.e. $\xi$. Thus we can continue our derivation:
        \begin{align*}
            \int_{\infty}^\infty (T_ng)\hat{\;}(\xi)\overline{(T_mg)\hat{\;}(\xi)} d\xi \\
            = \int_{-\infty}^\infty e^{-2\pi in\xi}\hat{g}(\xi)\overline{e^{-2\pi im\xi}\hat{g}(\xi)} d\xi && \text{(Identity for transform of translation)} \\
            = \int_{-\infty}^\infty e^{-2\pi in\xi}\overline{e^{-2\pi im\xi}}\hat{g}(\xi)\overline{\hat{g}(\xi)} d\xi && \text{(Modulus is multiplicative)} \\
            = \int_{-\infty}^\infty e^{-2\pi i(n - m)\xi}|\hat{g}(\xi)|^2 d\xi && (z\overline{z} = |z|^2, \overline{e^{-2\pi im\xi}} = e^{2\pi im\xi}) \\
            = \sum_{k \in \mathbb{Z}} \int_{k}^{k + 1} e^{-2\pi i(n - m)\xi}|\hat{g}(\xi)|^2 d\xi \\
            = \sum_{k \in \mathbb{Z}} \int_0^1 e^{{-2\pi i}(n - m)(\xi + k)}|\hat{g}(\xi + k)|^2 d\xi && \text{(Translating the integral)}
        \end{align*} 
        We would like now to interchange the sum and integral. This is possible as a special case of Fubini's theorem, with respect to the counting measure on $\mathbb{Z}$ and Lebesgue measure on $\mathbb{R}$. \\ \\
        In particular, refer temporarily to $f_{n,m}$ as the integrand. Then we have the following, as our exponenial always falls on the unit circle in the complex plane:
        $$\int_0^1 \sum_{k \in \mathbb{Z}} |f_{n,m}| = \int_0^1 \sum_{k \in \mathbb{Z}} |\hat{g}(\xi + k)|^2 = \int_{k}^{k + 1} \sum_{k \in \mathbb{Z}} |\hat{g}(\xi)|^2,$$
        where monotone convergence, given the non-negativity of each $|\hat{g}(\xi)|^2$, has 
        $$\int_{k}^{k + 1}\sum_{k \in \mathbb{Z}} |\hat{g}(\xi)|^2 = \sum_{k \in \mathbb{Z}} \int_{k}^{k + 1} |\hat{g}(\xi)|^2 = \int_{\infty}^\infty |\hat{g}(\xi)|^2 < \infty,$$
        where the finiteness of the last term follows as $g \in L^2(\mathbb{R}) \rightarrow \hat{g} \in L^2(\mathbb{R})$. \\ \\
        Thus $\int_0^1 \sum_{k \in \mathbb{Z}} |f_{n,m}|$ is finite; by Tonelli's theorem then, again with respect to the counting and Lebesgue measures, $\sum_{k \in \mathbb{Z}} \int_0^1 |f_{n,m}|$ is as well. Thus the hypotheses for Fubini's theorem are fulfilled, and we interchange the sum. We continue then:
        \begin{align*}
            \sum_{k \in \mathbb{Z}} \int_0^1 e^{{-2\pi i}(n - m)(\xi + k)}|\hat{g}(\xi + k)|^2 d\xi \\
            = \int_0^1 \sum_{k \in \mathbb{Z}} e^{{-2\pi i}(n - m)(\xi + k)}|\hat{g}(\xi + k)|^2 d\xi && \text{(Fubini's theorem)} \\
            = \int_0^1 \sum_{k \in \mathbb{Z}} e^{-2\pi i(n - m)\xi}|\hat{g}(\xi + k)|^2 d\xi \\
            = \int_0^1 \Big ( \sum_{k \in \mathbb{Z}} |\hat{g}(\xi + k)|^2 \Big)e^{-2\pi i(n - m)\xi} d\xi
        \end{align*}
        Define $H(\xi) = \sum_{k \in \mathbb{Z}} |\hat{g}(\xi + k)|^2$. Note then our earlier work done to show we can apply Fubini's theorem has $H \in L^1(\mathbb{R})$. \\ \\
        We also observe the $1$-periodicity of $\sum_{k \in \mathbb{Z}} |\hat{g}(\xi + k)|^2$, which is obvious as $k$ varies over all $\mathbb{Z}$. \\ \\
        Thus $H(\xi) = \sum_{k \in \mathbb{Z}} |\hat{g}(\xi + k)|^2 \in L^1(\mathbb{T})$; so its Fourier transform is defined. We can continue then:
        \begin{align*}
            \int_0^1 \Big ( \sum_{k \in \mathbb{Z}} |\hat{g}(\xi + k)|^2 \Big)e^{-2\pi i(n - m)\xi} d\xi \\
            = \int_0^1 H(\xi)e^{-2\pi i(n - m)\xi}d\xi \\
            = \hat{H}(n -m)
        \end{align*}
        Which is essentially the final fact we need. Assume then $\{T_kg\}_{k \in \mathbb{Z}}$ is an orthonormal sequence; then by our work we get the Fourier coefficients of $H$ as 
        \begin{equation*}\hat{H}(i) = \begin{cases}
            1 \quad \text{ if } i = 0 \\
            0 \quad \text{ if } i \neq 0 
        \end{cases} = \delta(i), \end{equation*}
        where $\delta$ is the Dirac delta function. Using the uniqueness of Fourier coefficients then, we must have $H = 1 \text{ a.e.}$, as the Dirac delta function gives precisely the Fourier coefficients of $1$. \\ \\
        Thus we have $H = 1 \text{ a.e.}$, which is precisely $\sum_{k \in \mathbb{Z}} |\hat{g}(\xi + k)|^2 = 1 \text{ a.e.}$ \\ \\
        Assume then instead we have $\sum_{k \in \mathbb{Z}} |\hat{g}(\xi + k)|^2 = 1 \text{ a.e.}$; we recall in the process of our derivation we also established
        $$\langle T_ng, T_mg \rangle = \int_0^1 \Big ( \sum_{k \in \mathbb{Z}} |\hat{g}(\xi + k)|^2 \Big)e^{-2\pi i(n - m)\xi} d\xi,$$
        so under our assumption this clearly just becomes 
        $$\langle T_ng, T_mg \rangle = \int_0^1 e^{-2\pi i(n - m)\xi}.$$
        When $n = m$, the integrand is $1$, so the integral reduces to $1$. When $n \neq m$, we can apply the FTC (note the integrand is continuous) to get
        \begin{align*}\int_0^1 e^{-2\pi i(n - m)\xi} = \frac{-e^{-2\pi i (n - m)}}{2 \pi i (n -m)} - \Bigg ( - \frac{1}{2\pi i (n-m)} \Bigg) \\
            - \frac{1}{2\pi i (n-m)} - \Bigg ( - \frac{1}{2\pi i (n-m)} \Bigg ) = 0,
        \end{align*}
        which in tandem with the previous fact clearly shows that $\{T_kg\}_{k \in \mathbb{Z}}$ is an orthonormal sequence. Thus we have shown both directions, and we are done.
    \end{proof}
\end{ex}

\begin{ex}{14}
    Let $p(x) = \mathbbm{1}_{[0,1)}(x)$ and $h(x) = \mathbbm{1}_{[0, \frac{1}{2})} -\mathbbm{1}_{[\frac{1}{2}, 1)}$. For $j, k \in \mathbb{Z}$, set $I_{j,k} = [2^{-j}k, 2^{-j}(k + 1))$ and define 
    \begin{equation*}
        \begin{cases}
            p_{j,k}(x) = 2^{\frac{j}{2}}p(2^jx - k) \\
            h_{j,k}(x) = 2^{\frac{j}{2}}h(2^jx - k)
        \end{cases}
    \end{equation*}
    \begin{enumerate}[label=(\alph*)]
        \item Prove that $\{h_{j,k}\}_{j, k \in \mathbb{Z}}$ is an orthonormal sequence in $L^2(\mathbb{R})$.
        \begin{proof}
            Let $j, k, j', k' \in \mathbb{Z}$ and consider $h_{jk}$ and $h_{j'k'}$. By the definition of $h_{jk}$, we can write 
            $$h_{j,k}(x) = 2^{\frac{j}{2}}(\mathbbm{1}_{[0, \frac{1}{2})}(2^jx - k) - \mathbbm{1}_{[\frac{1}{2}, 1)}(2^jx - k)).$$
            We inspect $\mathbbm{1}_{[0, \frac{1}{2})}(2^jx - k)$ first, note this function precisely is 
            \begin{equation*}
                \mathbbm{1}_{[0, \frac{1}{2})}(2^jx - k) = 
                \begin{cases}
                    1 & \text{if } 0 \leq 2^jx - k < \frac{1}{2} \\
                    0 & \text{otherwise}
                \end{cases},
            \end{equation*}
            where we note the following inequalities are equivalent:
            \begin{align*}
                0 \leq 2^j x - k < \frac{1}{2} \\
                k \leq 2^j x < \frac{1}{2} + k \\
                2^{-j}k \leq x < 2^{-j - 1} + 2^{-j}k \\
                2^{-j - 1}(2k) \leq x < 2^{-j - 1}(2k + 1),
            \end{align*}
            i.e. that $x \in I_{j + 1, 2k}$. Thus this function reduces to $\mathbbm{1}_{j + 1, 2k}$. Performing a similar analysis on the other function then, we note we can write
            $$h_{j,k} = 2^{\frac{j}{2}} (\mathbbm{1}_{I_{j + 1, 2k}} - \mathbbm{1}_{I_{j + 1, 2k + 1}}),$$
            with the same for $h_{j'k'}$. We evaluate the real inner product then:
            $$\langle h_{j,k}, h_{j',k'} \rangle = \int_{-\infty}^\infty h_{j,k}h_{j',k'} = \int_{-\infty}^\infty 2^{\frac{j}{2}} (\mathbbm{1}_{I_{j + 1, 2k}} - \mathbbm{1}_{I_{j + 1, 2k + 1}})2^{\frac{j'}{2}} (\mathbbm{1}_{I_{j' + 1, 2k'}} - \mathbbm{1}_{I_{j' + 1, 2k' + 1}})$$
            We can expand the integrand (ignoring the power of $2$) to get 
            \begin{align*}
                (\mathbbm{1}_{I_{j + 1, 2k}} - \mathbbm{1}_{I_{j + 1, 2k + 1}})(\mathbbm{1}_{I_{j' + 1, 2k'}} - \mathbbm{1}_{I_{j' + 1, 2k' + 1}}) \\ = \mathbbm{1}_{I_{j + 1, 2k}}\mathbbm{1}_{I_{j' + 1, 2k'}} - \mathbbm{1}_{I_{j + 1, 2k}}\mathbbm{1}_{I_{j' + 1, 2k' + 1}} - \mathbbm{1}_{I_{j + 1, 2k + 1}}\mathbbm{1}_{I_{j' + 1, 2k'}} + \mathbbm{1}_{I_{j + 1, 2k + 1}}\mathbbm{1}_{I_{j' + 1, 2k' + 1}}
            \end{align*}
            Say then first that $k = k'$ and $j = j'$. Then this reduces to 
            $$\mathbbm{1}_{I_{j + 1, 2k}} + \mathbbm{1}_{I_{j' + 1, 2k + 1}}$$
            as the two middle terms vanish as we have that $I_{n,k}$ and $I_{n,k + 1}$ are disjoint for any particular $n \in \mathbb{Z}$ by definition. Additionally, the square of an indicator function is itself. \\ \\  
            Integrating (adding back in the power of 2, note here $2^\frac{j}{2}2^\frac{j'}{2} = 2^\frac{j}{2}2^\frac{j}{2} = 2^j$), we note
            $$2^j\int_{-\infty}^\infty \mathbbm{1}_{I_{j + 1, 2k}} + \mathbbm{1}_{I_{j + 1, 2k + 1}} = 2^j(2^{-j - 1} + 2^{-j - 1}) = 1;$$
            this is the integral precisely because the two indicator function indicate disjoint intervals, where the length of both is intervals is $2^{-j - 1}$. \\ \\
            Suppose then that we have $k \neq k'$. Then the first and last product of indicator functions vanish, as they indicate intervals which are disjoint, i.e. we reduce to
            $$- \mathbbm{1}_{I_{j + 1, 2k}}\mathbbm{1}_{I_{j + 1, 2k' + 1}} - \mathbbm{1}_{I_{j + 1, 2k + 1}}\mathbbm{1}_{I_{j + 1, 2k'}},$$
            still under the assumption $j = j'$. \\ \\ We note then each product vanishes as long as we don't have $2k = 2k' + 1$ or $2k' = 2k + 1$ (as then the intervals are disjoint in the products), but this is automatically true as we have no integer that is even \textit{and} odd. \\ \\ 
            Thus this the integral evaluates to $0$ in this case. \\ \\
            Say then $j \neq j'$. We assume without loss of generality that $j < j'$, i.e. $j + 1 < j' + 1$ such that $2^{-j - 1} > 2^{- j' - 1}$. This assumption allows us to "nest" dyadic intervals, i.e. it guarentees the existence of some $I_{j + 1,n}$ in which $I_{j' +1, 2k'}$ and $I_{j' + 1, 2k' + 1}$ are properly contained. \\ \\This fact is obvious if we constructing dyadic intervals iteratively for each value $j \in \mathbb{Z}$. If one wishes for explicit proof, we just note that we have by definition
            $$I_{j' + 1, 2k'} \cup I_{j' + 1, 2k' + 1} = I_{j', k'};$$
            this of course is true precisely because the following are equivalent:
            \begin{align*}
                x \in I_{j' + 1, 2k'} \cup I_{j' + 1, 2k' + 1} \\
                2^{-j' - 1}(2k') \leq x < 2^{-j' - 1}(2k' + 1) \text{ or } 2^{-j' - 1}(2k' + 1) \leq x < 2^{-j' - 1}(2k' + 2) \\
                2^{-j'}k' \leq x < 2^{-j}(k' + 1) \\
                x \in I_{j', k'}
            \end{align*}
            Continuing this then, we can say something like
            $$I_{j', k'} \cup I_{j', k' + 1} = I_{j' - 1, n_1}$$
            when $k'$ is even, i.e. $k' = 2n_1$. If $k'$ is odd, the same result holds by viewing $k' = 2n_1 + 1$ and taking the second interval as $I_{j', k' - 1}$.\\ \\
            As $j < j'$ then, we can iteratively repeat this process $i$ many times until $j' - i = j$, where then we will have a chain of unions having the desired containment in $I_{j,n}$ for some $n$. \\ \\
            Moreover, it is clear this $n$ is unique, as $I_{j, m}$ is disjoint with $I_{j,n}$ for $n \neq m$. \\ \\
            A consequence of this is that as we get $j + 1 < j' + 1$, it is either that 
            $$I_{j' + 1, 2k'}, I_{j' + 1, 2k' + 1} \subseteq I_{j + 1, 2k},$$
            or neither is contained at all. Similarly, we have either 
            $$I_{j' + 1, 2k'}, I_{j' + 1, 2k' + 1} \subseteq I_{j + 1, 2k + 1},$$
            or neither is contained. Additionally, if we have that these two intervals were contained in $I_{j + 1, 2k}$, then they couldn't be contained in $I_{j + 1, 2k+1}$, and vice versa. \\ \\
            Notice then that we can recast the integrand (sans the power of 2) as 
            $$\mathbbm{1}_{I_{j + 1, 2k}} (\mathbbm{1}_{I_{j' + 1, 2k'}} - \mathbbm{1}_{I_{j' + 1, 2k' + 1}}) + \mathbbm{1}_{I_{j + 1, 2k + 1}}(\mathbbm{1}_{I_{j' + 1, 2k'}} - \mathbbm{1}_{I_{j' + 1, 2k' + 1}}).$$
            Our previous reasoning regarding the containment of intervals thus has that at least one of these terms vanishes (again, as containment in $I_{j + 1, 2k}$ or $I_{j + 1, 2k + 1}$ is mutually exclusive). \\ \\
            If both vanish, we are done. If one does not vanish then, our reasoning has that it must be (i.e. if the first term is the sole survivor) that 
            $$I_{j' + 1, 2k'}, I_{j' + 1, 2k' + 1} \subseteq I_{j + 1, 2k}.$$
            Then, inspecting the integral, we get 
            $$\int_{-\infty}^\infty \mathbbm{1}_{I_{j + 1, 2k}} (\mathbbm{1}_{I_{j' + 1, 2k'}} - \mathbbm{1}_{I_{j' + 1, 2k' + 1}}) = \int_{-\infty}^\infty \mathbbm{1}_{I_{j' + 1, 2k'}} - \mathbbm{1}_{I_{j' + 1, 2k' + 1}};$$
            where there intervals both have length $2^{-(j' + 1)}$ and are necessarily disjoint. Thus as we have opposite sign of the indicators, the integral vanishes. \\ \\
            We have consequently exhaustively completed all cases, so we can conclude $\{h_{j,k}\}_{j, k \in \mathbb{Z}}$ is an orthnormal sequence in $L^2(\mathbb{R})$ as expected.




        \end{proof}
        \item For $j \in \mathbb{Z}$, show that $\{p_{jk}\}_{k \in \mathbb{Z}}$ is an orthnormal sequence in $L^2(\mathbb{R})$. 
        \begin{proof}
            Consider some fixed $j \in \mathbb{Z}$. We can perform a similar ``trick" to the previous part, i.e. we can express (much more obviously)
            $$p_{j,k} = 2^{\frac{j}{2}}\mathbbm{1}_{I_{j,k}}$$
            given that $2^jx - k$ sends $2^{-j}k$ to $0$ and $2^{j}k + 1$ to $1$, i.e. it linearly interpolates $[-2^{-j}k, 2^{-j}k + 1)$ onto $[0,1)$. \\ \\
            We can view the inner product $\langle p_{jk}, p_{jk'} \rangle$ then. Expanding we get 
            $$\langle p_{jk}, p_{jk'} \rangle = \int_{-\infty}^\infty 2^{\frac{j}{2}}\mathbbm{1}_{I_{j,k}}2^{\frac{j}{2}}\mathbbm{1}_{I_{j,k'}} = 2^j\int_{-\infty}^\infty \mathbbm{1}_{I_{j,k}}\mathbbm{1}_{I_{j,k'}}$$
            As observed earlier, we only have that the indicated intervals are \textit{not} disjoint if $k = k'$, i.e. the integral vanishes when $k \neq k'$.  \\ \\
            If instead $k = k'$, then the product is just $\mathbbm{1}_{j,k}$, an interval of length $2^{-j}$. In this case then 
            $$2^j\int_{-\infty}^\infty \mathbbm{1}_{I_{j,k}} = 2^j\int_{I_{j,k}} \mathbbm{1}_{j,k} = 2^j2^{-j} = 1.$$
            Taking these two facts together shows $\{p_{jk}\}_{k \in \mathbb{Z}}$ is an orthnormal sequence.
        \end{proof}
        \item Fix $j \in \mathbb{Z}$ and let $g_j(x)$ be a step function, constant on all dyadic interals $I_{j,k}$ for $k \in \mathbb{Z}$. Show that $g_j(x) = g_{j - 1}(x) + r_{j - 1}(x)$ where 
        $$r_{j - 1}(x) = \sum_{k \in \mathbb{Z}} a_{j - 1}(k)h_{j - 1, k}(x)$$
        for some coefficients $\{a_{j - 1}(k)\}_{k \in \mathbb{Z}}$ and $g_{j - 1}(x)$ a step function constant on all intervals $I_{j - 1, k}$ for $k \in \mathbb{Z}$.
        \begin{proof}
            Consider some $j \in \mathbb{Z}$ and the corresponding $g_j$. Note first we can decompose $I_{j - 1, k}$ as $I_{j, 2k} \cup I_{j, 2k + 1}$ as proven earlier. \\ \\
            Let then $n = g_j(I_{j, 2k})$ and $m = g_j(I_{j, 2k + 1})$. Take $g_{j - 1}(x) = \frac{n + m}{2}$ for $x \in I_{j - 1,k}$ and stipulate $a_{j - 1}(k) = 2^{\frac{1 - j}{2}}\frac{n - m}{2}$. It follows for $x \in I_{j,2k}$ that
            \begin{align*}g_{j - 1}(x) + r_{j -1}(x) \\ = \frac{n + m}{2} + \sum_{i \in \mathbb{Z}} a_{j - 1}(i)h_{j-1,i}(x) \\ = \frac{n + m}{2} + a_{j - 1}(k)h_{j-1,k}(x) \end{align*}
            Where the $k$th term in the sum is clearly the only one to survive as we have the earlier identity
            $$h_{j - 1, k}(x) = 2^\frac{j-1}{2} (\mathbbm{1}_{I_{j, 2k}}(x) - \mathbbm{1}_{I_{j, 2k + 1}}(x)),$$
            and so we can continue
            \begin{align*}\frac{n + m}{2} + a_{j - 1}(k)h_{j-1,k}(x) \\ = \frac{n + m}{2} + \frac{n - m}{2}2^{\frac{1 - j}{2}}2^\frac{j - 1}{2} \\ = \frac{n + m}{2} + \frac{n - m}{2} = n = g_j(x) \end{align*}
            where we have a symmetric argument to get that $g_{j - 1}(x) + r_{j - 1}(x) = m$ for $x \in I_{j, 2k + 1}$; note all that changes is the sign of $h_{j - 1,k}(x)$, which gives the desired difference $\frac{n + m}{2} - \frac{n - m}{2} = m$. \\ \\
            Thus on $I_{j - 1, k}$, we have $g_{j - 1} + r_{j - 1} = g_j$. We simply duplicate this construction for all other $k \in \mathbb{Z}$ to get the result in general; this also has that the resultant $g_{j - 1}$ is constant on each $I_{j -1, k}$ for $k \in \mathbb{Z}$.
        \end{proof}
        \item Now fix $J \geq 0$ and consider 
        $$\{p_{J,k} : 0 \leq k \leq 2^J - 1\} \cup \{h_{j, k} : j \geq J \text{ and } 0 \leq k \leq 2^j - 1\}.$$
        Prove that this expression is an orthonormal sequence in $L^2[0,1]$.
        \begin{proof}
            Note, as already proven, the former terms in the expression are orthonormal with respect to $k \in \mathbb{Z}$, as are the latter with respect to $j, k \in \mathbb{Z}$. \\ \\
            It follows all that we need to verify is that $\langle p_{J,k}, h_{j,k'} \rangle = 0$ for $j \geq J, 0 \leq k \leq 2^J - 1, 0 \leq k' \leq 2^j - 1$. \\ \\
            Consider then the following computation:
            \begin{align*}
                \langle p_{J,k}, h_{j, k'} \rangle \\
                = \int_{-\infty}^\infty p_{J,k}h_{j,k'} \\
                = \int_{-\infty}^\infty 2^{\frac{J}{2}}\mathbbm{1}_{I_{J,k}} \Big (2^{\frac{j}{2}}(\mathbbm{1}_{I_{j + 1, 2k'}} + \mathbbm{1}_{I_{j + 1, 2k' + 1}}) \Big) \\
                = 2^\frac{J + j}{2} \int_{-\infty}^\infty \mathbbm{1}_{I_{J, k}}\mathbbm{1}_{I_{j + 1, 2k'}} - \mathbbm{1}_{I_{J,k}}\mathbbm{1}_{I_{j + 1, 2k' + 1}}
            \end{align*}
            Recall then the argumet we presented in part (a), i.e. that dyadic intervals nest. \\ \\
            In particular, as $j + 1 > J$, it is either that we have both $I_{j + 1, 2k'}$ and $I_{j + 1, 2k' + 1}$ properly contained in $I_{J,k}$, or neither is. \\ \\
            In the latter case, both products vanish, so the integral evaluates to 0 as desired. \\ \\
            In the former case, we have proper containment, so the first term is $1$ on $I_{j + 1, 2k'}$, and the the second is $-1$ on $I_{j + 1, 2k' + 1}$. Then as both these intervals are the same length (and disjoint), we have that the integral evaluates to $1 - 1 = 0$. \\ \\
            Thus we have verified  $\langle p_{J,k}, h_{j,k'} \rangle = 0$ for $j \geq J, 0 \leq k \leq 2^J - 1, 0 \leq k' \leq 2^j - 1$, which is sufficient to conclude that $\{p_{J,k} : 0 \leq k \leq 2^J - 1\} \cup \{h_{j, k} : j \geq J \text{ and } 0 \leq k \leq 2^j - 1\}$ is an orthnormal sequence.

        \end{proof}
        \item For $f \in L^2[0,1]$, find $g_j (j \geq J)$ a step function constant on all $I_{j,k}$ for $k \in \mathbb{Z}$ that approximates $f$ in the $L^2$ norm. Use this and the previous question to show the system defined in the previous part is an orthonormal basis for $L^2[0,1]$.
        \begin{proof}
            Recall $C[0,1] \cap L^2[0,1]$ is dense in $L^2[0,1]$. We assume these spaces are real-valued to start, and consider $f \in C[0,1] \cap L^2[0,1]$. \\ \\
            Define then $g_J$ by stipulating 
            $$g_J = \sum_{k = 0}^{2^J - 1} \langle f, p_{J,k} \rangle p_{J,k}.$$
            For $j \geq J$, define $g_{j + 1} = g_j + \sum_{k = 0}^{2^j - 1} \langle f, h_{j,k}\rangle h_{j,k}$. Note then 
            $$\langle f, p_{J,k} \rangle = \int_{[0,1]} fp_{J,k} = \int_{I_{J, k}} f2^{\frac{J}{2}}.$$
            We recall this integral is finite as we have the inclusion $L^2[0,1] \subseteq L^1[0,1]$ via application of H\"older's inequality. \\ \\
            Consider then $x \in I_{J,k}$. We get 
            $$g_J(x) = \sum_{k = 0}^{2^J - 1} \langle f, p_{J,i} \rangle p_{J,i}(x) = \langle f, p_{J,k} \rangle p_{J,k}(x) = 2^{\frac{J}{2}}\int_{I_{J, k}} f2^{\frac{J}{2}} = 2^J \int_{I_{J,k}} f.$$
            Recall then that $|I_{J,k}| = 2^{-J}$, so this is in particular that $g_J$ on $I_{J,k}$ takes on precisely the average value of $f$ over $I_{J,k}$. \\ \\
            We look now at the process for $g_{J + 1}$, calculating the inner product 
            $$\langle f, h_{J,k} \rangle = \int_0^1 f2^{\frac{J}{2}}(\mathbbm{1}_{I_{J + 1, 2k}} - \mathbbm{1}_{I_{J + 1, 2k + 1}}) = 2^{\frac{J}{2}} \Big ( \int_{I_{J + 1, 2k}} f - \int_{I_{J + 1, 2k + 1}} f \Big)$$
            Letting then $x \in I_{J + 1,k}$ then, we get 
            $$g_{J + 1}(x) = \langle f, p_{J, \lfloor \frac{k}{2} \rfloor}\rangle p_{J, \lfloor \frac{k}{2} \rfloor}(x) + \langle f, h_{J, \lfloor \frac{k}{2} \rfloor}\rangle h_{J, \lfloor \frac{k}{2} \rfloor}(x).$$
            When $k$ is even, this reduces nicely to 
            $$2^{\frac{J}{2}}(\langle f, p_{J,\lfloor \frac{k}{2} \rfloor} \rangle + \langle f, h_{J,\lfloor \frac{k}{2} \rfloor} \rangle),$$
            whereas when $k$ is odd we get
            $$2^{\frac{J}{2}}(\langle f, p_{J,\lfloor \frac{k}{2} \rfloor} \rangle - \langle f, h_{J,\lfloor \frac{k}{2} \rfloor} \rangle).$$
            Looking at the former case, we get 
            \begin{align*}g_{J + 1}(x) = 2^J \Big (\int_{I_{J, \lfloor \frac{k}{2} \rfloor}} f + \int_{I_{J + 1, k}} f - \int_{I_{J + 1, k + 1}} f \Big) \\
                 = 2^J \Big (\int_{I_{J + 1, k}} f + \int_{I_{J + 1, k + 1}} f + \int_{I_{J + 1, k}} f - \int_{I_{J + 1, k + 1}} \Big) = 2^{J + 1}\int_{I_{J + 1, k}} f,
            \end{align*}
            which is precisely that $g_{J + 1}$ on $I_{J + 1,k}$ takes on the average value of $f$ on $I_{J + 1, k}$. \\ \\
            In principle, we induct on this construction. In iterating this computation, we get $g_{J + n}$ takes on the average value of $f$ on each $I_{J + n, k}$ as we can think of each $h_{J + n, k}$ as a ``difference of averages", i.e. with respect to the $I_{J + n + 1, 2k}$ and $I_{J + n + 1, 2k + 1}$ intervals. \\ \\
            Thus a similar calculation has that $g_{J + n + 1}$ takes on the average value of $f$ over each $I_{J + n + 1, k}$ interval and so on. \\ \\
            Note then as we assumed $f$ to be continuous, and $[0,1]$ is compact, it is that $f$ is uniformly continuous. Let then $\epsilon > 0$. Then there is a $\delta > 0$ such that 
            $$\forall x, y \in [0,1], \; |x - y| < \delta \longrightarrow |f(x) - f(y)| < \epsilon$$
            Consider then the $g_j$, $j \geq J$, such that $2^{-j} < \delta$, i.e. the dyadic intervals all have length smaller than $\delta$. \\ \\
            Consider then $||f - g_j||_\infty$. Let $x \in [0,1]$, then $x \in I_{j, k}$ for some $k$. Recall
            $$g_r(x) = 2^J \int_{I_{J, k}} f,$$
            which is precisely the average of $f$ over the interval $I_{j,k}$. By the integral mean value theorem then, there is a $c \in I_{j, k}$ such that 
            $$f(c) = 2^J \int_{I_{J, k}} f = g_r(x).$$
            It follows
            $$|f(x) - g_r(x)| \leq |f(x) - f(c)| + |f(c) - g_r(x)| < \epsilon$$
            as $|x - d| < \delta$, given we took $I_{j,k}$ smaller than $\delta$. \\ \\
            As our choice of $x$ was arbitrary then, we get 
            $$||f - g_r||_\infty < \epsilon,$$
            i.e. we have $g_r$ converges to $f$ uniformly as $r \rightarrow \infty$. Thus $g_r$ also converges to $f$ in the $L^2$ norm, given $[0,1]$ is finite measure. \\ \\
            So we have that we can approximate $f$ with these sort of step functions. \\ \\As we can approximate any $L^2[0,1]$ function with a continuous function then, the triangle inequality has that we can approximate $L^2[0,1]$ functions with these step functions. \\ \\
            As each $g_r$ is just a sum of scaled $h_{j,k}$s and $p_{J,k}$s in $\{p_{J,k} : 0 \leq k \leq 2^J - 1\} \cup \{h_{j, k} : j \geq J \text{ and } 0 \leq k \leq 2^j - 1\}$ then, this demonstrates this collection is an orthonormal basis. \\ \\
            Recall then we assumed the our spaces were real-valued; note for this sake the complex case follows by splitting the complex-valued function into its real and complex parts.
        \end{proof}
    \end{enumerate}
\end{ex}

\end{document}
