\documentclass[12pt]{article}

\usepackage[margin=1in]{geometry} 
\usepackage{amsmath,amsthm,amssymb,enumitem}

\newcommand{\N}{\mathbb{N}}
\newcommand{\Z}{\mathbb{Z}}
\newcommand{\R}{\mathbb{R}}
\newcommand{\Rd}{\mathbb{R}^{d}}
\newcommand{\exr}{[-\infty, \infty]}
\newcommand{\bignorm}{\Big | \Big |}

\newenvironment{ex}[2][Exercise]{\begin{trivlist}
\item[\hskip \labelsep {\bfseries #1}\hskip \labelsep {\bfseries #2.}]}{\end{trivlist}}

\newenvironment{sol}[1][Solution]{\begin{trivlist}
\item[\hskip \labelsep {\bfseries #1:}]}{\end{trivlist}}

\newenvironment{theorem}[2][Theorem]{\begin{trivlist}
\item[\hskip \labelsep {\bfseries #1}\hskip \labelsep {\bfseries #2.}]}{\end{trivlist}}

\newenvironment{definition}[2][Definition]{\begin{trivlist}
\item[\hskip \labelsep {\bfseries #1}\hskip \labelsep {\bfseries #2.}]}{\end{trivlist}}
        
\newenvironment{example}[1][Example]{\begin{trivlist}
\item[\hskip \labelsep {\bfseries #1:}]}{\end{trivlist}}


\begin{document}
\noindent David Owen Horace Cutler \hfill {\Large Math 237: Lecture 2} \hfill \today

\begin{theorem}{(Parallelogram Law)}
    If $H$ is an inner product space, then for every $x, y \in H$:
    \begin{equation}
        ||x + y||^2 + ||x - y||^2 + 2(||x||^2 + ||y||^2)
    \end{equation}
\end{theorem}

\begin{example}{(lp Norm)}
    We define the $l_p$ norm for $\mathbb{C}^d$ for $\forall p \in (0, \infty)$ by the following convention:
    \begin{equation}
        ||x||_p = \Big ( \sum_{k = 1}^d |x_k|^p \Big )^{\frac{1}{p}}
    \end{equation}
    Interestingly, $p = 2$ uniquely generates the norm that satisifes the Parallelogram law. 
\end{example}

\begin{theorem}{(Unique Closest Point)}
    Let $H$ be a Hilbert space and $M$ be a closed subspace of $H$. \\ \\
    For every $x \in H$, there exists a unique $p \in M$ that is closest to $x$. That is that for some $p \in M$:
    \begin{equation}
        ||x - p|| = \inf \{||x - m|| , m \in M\} = \text{dist}(x, M)
    \end{equation}
    \begin{proof}
        Let $d = \text{dist}(x, M)$. By the definition of the infinum, $\{y_n\}_{n = 1}^\infty \subseteq M$ such that $d = \underset{n \rightarrow \infty}{\lim} ||x - y_n||$. \\ \\
        Take then $d^2 = \underset{n \rightarrow \infty}{\lim} ||x - y_n||^2$. Let $\epsilon > 0$. Using the definition of the limit, there $\exists N \in \mathbb{N}$ such that $\forall n \geq N$:
        $$d^2 \leq ||x - y_n||^2 < d^2 + \epsilon$$
        Take then $n, m \geq N$, this yields the following by applying the Parallelogram law:
        \begin{equation}
                ||(x - y_n) - (x - y_m)||^2 + ||(x - y_n) + (x + y_m)||^2 = 2(||x - y_n||^2 + ||x - y_m||^2)
        \end{equation}
        However, we also get the following by using homogeneity:
        \begin{equation}
            ||(x - y_n) - (x - y_m)||^2 + ||(x - y_n) + (x + y_m)||^2 = ||y_n - y_m||^2 + 4||x - \frac{(y_n + y_n)}{4}||^2
        \end{equation}
        Which gets us the following (as $\frac{(y_n + y_m)}{2} \in M$):
        \begin{equation}
            \begin{aligned}
                4d^2 + ||y_n - y_m||^2 \leq 4d^2 + 4 \bignorm x - \frac{(y_n - y_m)}{2} \bignorm^2 \\
                = 2(||x - y_n||^2 + ||x - y_m||^2) = 2||x - y_n||^2 + 2||x - y_m||^2 < 4d^2 + 4\epsilon 
            \end{aligned}
        \end{equation}
        Which of course finally has:
        \begin{equation}
            ||y_n - y_m||^2 < \epsilon
        \end{equation}
        So $\{y_n\}_{n = 1}^\infty$ is Cauchy, and so it converges to some $y \in M$, where we get the final idea:
        \begin{equation}
            d = \underset{n \rightarrow \infty}{\lim} ||x - y_n|| = ||x - \underset{n \rightarrow \infty}{\lim} y_n || = ||x - y||
        \end{equation}
        Uniqueness is left as exercise to the reader.
    \end{proof}
\end{theorem}

\end{document}