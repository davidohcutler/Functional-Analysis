\documentclass[12pt]{article}

\usepackage[margin=1in]{geometry} 
\usepackage{amsmath,amsthm,amssymb,enumitem}

\newcommand{\N}{\mathbb{N}}
\newcommand{\Z}{\mathbb{Z}}
\newcommand{\R}{\mathbb{R}}
\newcommand{\Rd}{\mathbb{R}^{d}}
\newcommand{\exr}{[-\infty, \infty]}
\newcommand{\bignorm}{\Big | \Big |}

\newenvironment{ex}[2][Exercise]{\begin{trivlist}
\item[\hskip \labelsep {\bfseries #1}\hskip \labelsep {\bfseries #2.}]}{\end{trivlist}}

\newenvironment{sol}[1][Solution]{\begin{trivlist}
\item[\hskip \labelsep {\bfseries #1:}]}{\end{trivlist}}

\newenvironment{theorem}[2][Theorem]{\begin{trivlist}
\item[\hskip \labelsep {\bfseries #1}\hskip \labelsep {\bfseries #2.}]}{\end{trivlist}}

\newenvironment{definition}[2][Definition]{\begin{trivlist}
\item[\hskip \labelsep {\bfseries #1}\hskip \labelsep {\bfseries #2.}]}{\end{trivlist}}
        
\newenvironment{example}[1][Example]{\begin{trivlist}
\item[\hskip \labelsep {\bfseries #1:}]}{\end{trivlist}}


\begin{document}
\noindent David Owen Horace Cutler \hfill {\Large Math 237: Lecture 4} \hfill \today

\begin{theorem}{(Parallelogram Law)}
    If $H$ is an inner product space, then for every $x, y \in H$:
    \begin{equation}
        ||x + y||^2 + ||x - y||^2 + 2(||x||^2 + ||y||^2)
    \end{equation}
\end{theorem}

\begin{example}{(lp Norm)}
    We define the $l_p$ norm for $\mathbb{C}^d$ for $\forall p \in (0, \infty)$ by the following convention:
    \begin{equation}
        ||x||_p = \Big ( \sum_{k = 1}^d |x_k|^p \Big )^{\frac{1}{p}}
    \end{equation}
    Interestingly, $p = 2$ uniquely generates the norm that satisifes the Parallelogram law. 
\end{example}

\begin{theorem}{(Unique Closest Point)}
    Let $H$ be a Hilbert space and $M$ be a closed subspace of $H$. \\ \\
    For every $x \in H$, there exists a unique $p \in M$ that is closest to $x$. That is that for some $p \in M$:
    \begin{equation}
        ||x - p|| = \inf \{||x - m|| , m \in M\} = \text{dist}(x, M)
    \end{equation}
    \begin{proof}
        Let $d = \text{dist}(x, M)$. By the definition of the infinum, $\{y_n\}_{n = 1}^\infty \subseteq M$ such that $d = \underset{n \rightarrow \infty}{\lim} ||x - y_n||$. \\ \\
        Take then $d^2 = \underset{n \rightarrow \infty}{\lim} ||x - y_n||^2$. Let $\epsilon > 0$. Using the definition of the limit, there $\exists N \in \mathbb{N}$ such that $\forall n \geq N$:
        $$d^2 \leq ||x - y_n||^2 < d^2 + \epsilon$$
        Take then $n, m \geq N$, this yields the following by applying the Parallelogram law:
        \begin{equation}
                ||(x - y_n) - (x - y_m)||^2 + ||(x - y_n) + (x + y_m)||^2 = 2(||x - y_n||^2 + ||x - y_m||^2)
        \end{equation}
        However, we also get the following by using homogeneity:
        \begin{equation}
            ||(x - y_n) - (x - y_m)||^2 + ||(x - y_n) + (x + y_m)||^2 = ||y_n - y_m||^2 + 4||x - \frac{(y_n + y_n)}{4}||^2
        \end{equation}
        Which gets us the following (as $\frac{(y_n + y_m)}{2} \in M$):
        \begin{equation}
            \begin{aligned}
                4d^2 + ||y_n - y_m||^2 \leq 4d^2 + 4 \bignorm x - \frac{(y_n - y_m)}{2} \bignorm^2 \\
                = 2(||x - y_n||^2 + ||x - y_m||^2) = 2||x - y_n||^2 + 2||x - y_m||^2 < 4d^2 + 4\epsilon 
            \end{aligned}
        \end{equation}
        Which of course finally has:
        \begin{equation}
            ||y_n - y_m||^2 < \epsilon
        \end{equation}
        So $\{y_n\}_{n = 1}^\infty$ is Cauchy, and so it converges to some $y \in M$, where we get the final idea:
        \begin{equation}
            d = \underset{n \rightarrow \infty}{\lim} ||x - y_n|| = ||x - \underset{n \rightarrow \infty}{\lim} y_n || = ||x - y||
        \end{equation}
        Uniqueness is left as exercise to the reader.
    \end{proof}
\end{theorem}

\begin{definition}{(Orthogonal Projection)}
    If $M$ is a closed subspace of a Hilbert space $H$, for every $x \in H$, the unique $p \in M : ||x - p|| = \text{dist}(x, M)$ is called the \textbf{orthogonal projection} of $x$ onot $M$ denoted $P_M x$. \\ \\
    Turning this into a function $P_M$, we note it is a surjection. Moreover $P_M \in B(H)$. 
\end{definition}

\begin{definition}{(Orthogonal Complement)}
    Let $A$ be a subset of an inner product space H, we define the following to be the \textbf{orthogonal complement of A}:
    \begin{equation}
        A^\perp = \{x \in H : \langle x , a \rangle = 0, \forall a \in A\}
    \end{equation}
    If $A \subseteq H$ is a subset and $H$ is an inner product space, then $A^\perp$ is a subspace of $H$.
\end{definition}

\begin{definition}{(Spans and Completeness)}
    Let $S$ be a subset of a normed space $H$. The \textbf{span} of $S$ is the subspace of $H$ that contains all finite linear combinations of vectors in $S$. \\ \\
    The \textbf{closed linear span} of $S$ is the closure of the linear span of $S$, denoted $\overline{\text{span}(S)}$. \\ \\
    $S$ is said to be \textbf{complete} if $\overline{\text{span}(S)} = H$.
\end{definition}

\begin{theorem}{(Orthogonal Decomposition)}
    Let $H$ be a Hilbert space, $M$ a closed subspace, and $x \in H$. Then, the following statements are equivalent:
    \begin{enumerate}[label=(\alph*)]
        \item $x = p + e$, where $p = P_M x$
        \item $x = p + e$ where $p \in M$ and $e \in M^\perp$
        \item $x = p + e$, where $e = P_{M^\perp} x$
    \end{enumerate}
\end{theorem}

\begin{theorem}{(Spans and Orthogonal Complements)}
    Let $H$ be a Hilbert space, and $A$ a subset of $H$.
    \begin{enumerate}[label=(\alph*)]
        \item If $M$ is a closed subspace of $H$:
        $$(M^\perp)^\perp = M$$
        \item If $A \subseteq H$:
        $$A^\perp = (\overline{\text{span}(A)})^\perp$$ and 
        $$(A^\perp)^\perp = ((\overline{\text{span}(A)})^\perp)^\perp = \overline{\text{span}(A)}$$
        \item If $A = \{y_n\}_{n = 1}^\infty$ is a sequence, $A$ is complete if and only $\forall x \in H$, $\langle x, y_n \rangle = 0, \forall n \in \mathbb{N}$.
    \end{enumerate}
\end{theorem}

\begin{definition}{(Basis)}
    Let $H$ be a Hilbert space. A sequence $\{x_n\}_{n = 1}^\infty$ is a \textbf{basis} for $H$ if $\forall x \in H$:
    $$x = \sum_{k = 1}^\infty c_kx_n$$
    For some unique constants $\{c_k\}_{k = 1}^\infty$.
\end{definition}

\begin{definition}{(Orthogonal Sequence)}
    A sequence $\{x_n\}_{n = 1}^\infty$ is called \textbf{orthogonal} if $\langle x_n, x_m \rangle = 0$ when $n \neq m$.
\end{definition}

\begin{definition}{(Orthonormal Sequence)}
    An \textbf{orthonormal} sequence is is an orthogonal sequence with the added behavior that:
    \begin{equation}
        \langle x_n, x_m \rangle = \delta_{n,m} = \begin{cases}
            1 & \text{if } n = m \\
            0 & \text{if } n \neq m 
        \end{cases}
    \end{equation}
\end{definition}

\begin{definition}{(Orthonormal Basis)}
    A sequence $\{x_n\}_{n = 1}^\infty$ is an \textbf{orthonormal basis} if it is both orthonormal and a basis.
\end{definition}

\begin{theorem}{(Big Theorem on Orthnormal Sequences)}
    If $\{x_n\}_{n = 1}^\infty$ is an orthonormal sequence in a Hilbert space $H$, then:
    \begin{enumerate}[label=(\alph*)]
        \item \textbf{(Bessel's Inequality)} $\forall x \in H, \sum_{n = 1}^\infty |\langle x, x_n \rangle|^2 \leq ||x||^2$
        \item If $x = \sum_{n = 1}^\infty cx_n$ for a sequence $\{c_n\}_{n = 1}^\infty \subseteq F$, then $c_n = \langle x, x_n \rangle$ for all $n \geq 1$.
        \item $\sum_{n = 1}^\infty c_nx_n$ converges $\Leftrightarrow \sum_{n = 1}^\infty |c_n|^2 < \infty$
        \item $x \in \overline{\text{span}\{x_n\}_{n = 1}^\infty} \Leftrightarrow x = \sum_{n = 1}^\infty \langle x, x_n \rangle x_n$
        \item For $x \in H, p = \sum_{n = 1}^\infty \langle x, x_n \rangle x_n$ is the orthogonal projection of $x$ onto $\overline{\text{span}\{x_n\}_{n = 1}^\infty}$.
    \end{enumerate}
    \begin{proof}
        Proof in next lecture's notes.
    \end{proof}
\end{theorem}

\end{document}